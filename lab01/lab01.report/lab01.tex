\documentclass[a4paper,UTF8]{ctexart}
\usepackage{graphicx}
\usepackage{geometry}
\usepackage{xcolor}
\usepackage{amsmath}
\usepackage{enumerate}
\usepackage{caption}
\usepackage{listings}
\captionsetup[lstlisting]{labelfont=bf,justification=justified}
\usepackage{tikz}
\usepackage{pgfplots}
\pgfplotsset{compat=1.17}
\usepackage{appendix}

\graphicspath{{img/}}

\usepackage[colorlinks,linkcolor=blue]{hyperref}
\usepackage{bookmark}
\providecommand{\code}[2]{\lstinputlisting[language=#2,caption=\href{run:#1}{\ttfamily #1}]{#1}}
\providecommand{\img}[1]{\includegraphics[width=0.88\textwidth]{#1}}

% listings
\definecolor{grey}{rgb}{0.8,0.8,0.8}
\definecolor{darkgreen}{rgb}{0,0.3,0}
\definecolor{darkblue}{rgb}{0,0,0.3}
\lstset{%
    numbers=left, %行号
    numberstyle=\tiny\color{grey},
    showstringspaces=false,
    showspaces=false,%
    tabsize=4,%
    frame=shadowbox,%
    basicstyle={\ttfamily\scriptsize},%
    keywordstyle=\color{blue!80!black}\bfseries,%
    identifierstyle=,%
    commentstyle=\color{green!50!blue}\itshape,%
    stringstyle=\color{green!50!black},%
    rulesepcolor=\color{gray!20!white},
    breaklines,
    columns=flexible,
    extendedchars=false,
    %mathescape=true,
    language=verilog,
}

\begin{document}
\title{\normalsize \underline{计算机系统结构实验}\\\LARGE 实验 1 报告\\\vspace*{1em}\normalsize FPGA 基础实验 \\ LED Flow Water Light}
\author{李子龙\\ 518070910095}
\date{\today}
\maketitle
\tableofcontents
\clearpage

\section{实验目的}

\begin{enumerate}
    \item 熟悉 Xilinx 逻辑设计工具 Vivado 的基本操作
    \item 掌握使用 Verilog HDL 进行简单的逻辑设计
    \item 使用功能仿真与 I/O Planing 添加管脚约束
    \item 生成 Bitstream 文件并上板验证流水灯
\end{enumerate}

\section{实验原理与实现}

本实验通过需要实现 8 阶流水灯。首先使用 \verb"cnt_reg" 寄存器记录时钟周期数目:

\begin{lstlisting}
    reg [23:0] cnt_reg;
    always @ (posedge clock)
        begin
            if (reset)
                cnt_reg <= 0;
            else
                cnt_reg <= cnt_reg + 1;
        end
\end{lstlisting}

接着,对于 LED 灯所对应的循环,如果是复位信号,就会改为 \verb"01H",即仅点亮最低位的灯。当计数器循环满时,如果 LED 是 \verb"80H",即仅最高位灯亮,则改为 \verb"01H" 仅最低位亮以循环;否则就直接向左移位,以仅点亮下一个高位的灯。

\begin{lstlisting}
    reg [7:0] light_reg;
    always @(posedge clock) 
    begin
        if (reset) 
            light_reg <= 8'h01;
        else if (cnt_reg == 24'hffffff)
            begin
                if (light_reg==8'h80)
                    light_reg <= 8'h01;
                else
                    light_reg <= light_reg << 1;
            end
    end
\end{lstlisting}
当然在上板子后,对于该硬件 xc7k325tffg676-2 需要修改 \verb"reset" 为取反信号,最终实现请见附录 \ref{app:design}。

\appendix
\section{流水灯设计代码实现}\label{app:design}
\begin{lstlisting}[caption={flowing_light.v}]
`timescale 1ns / 1ps
//////////////////////////////////////////////////////////////////////////////////
// Company: SJTU CS
// Engineer: Zilong Li
// 
// Create Date: 2021/05/12 09:10:24
// Design Name: 
// Module Name: flowing_light
// Project Name: 
// Target Devices: 
// Tool Versions: 
// Description: 
// 
// Dependencies: 
// 
// Revision:
// Revision 0.01 - File Created
// Additional Comments:
// 
//////////////////////////////////////////////////////////////////////////////////

module flowing_light (
    input clock_p,
    input clock_n,
    input reset,
    output [7:0] led
);

    reg [23:0] cnt_reg;
    reg [7:0] light_reg;

    IBUFGDS IBUFGDS_inst (
        .O(CLK_i),
        .I(clock_p),
        .IB(clock_n)
    );

    always @ (posedge CLK_i)
        begin
            if (!reset)     // 板子电平取反
                cnt_reg <= 0;
            else
                cnt_reg <= cnt_reg + 1;
        end
    always @(posedge CLK_i) 
        begin
            if (!reset) 
                light_reg <= 8'h01;
            else if (cnt_reg == 24'hffffff)
                begin
                    if (light_reg==8'h80)
                        light_reg <= 8'h01;
                    else
                        light_reg <= light_reg << 1;
                end
        end
    assign led = light_reg;
endmodule
\end{lstlisting}

% DEPRECATED - NO REPORT REQURIED.

\end{document}
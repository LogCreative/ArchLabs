\begin{tikzpicture}
\ctikzset{multipoles/dipchip/width=2, multipoles/flipflop/width=2}
\tikzstyle{element} = [flipflop,text width=4.5em,text centered,font=\small\sffamily];
\node [ALU] at (4.5,4) {{\rotatebox{90}{\small \ttfamily ALU}}};
\node [element,] (PC) at (-9.5,-1.5) {PC};
\node [element,flipflop def={t1=readAddress,t5=Instruction[31:0]},text height=4.5em] (Inst) at (-4.5,-1.5) {Instruction Memory};
\draw (PC.east) to[short,-*] (-7,-1.5) -| (Inst.pin 1);
\draw (Inst.pin 5) to[short,-*] (-2.5,-1.5) to[short,-*] (-2,-1.5);

\node [dipchip, hide numbers, no topmark, external pins width=0,font=\small\sffamily] (Reg) at (2,-1.5) {Registers};
\node [right] at (Reg.bpin 1) {readReg1};
\node [right] at (Reg.bpin 2) {readReg2};
\node [right] at (Reg.bpin 3) {writeReg};
\node [right] at (Reg.bpin 4) {writeData};

\node [one bit adder,external pins width=0] (oALU) at (-5,5) {+};
\draw (-7,-1.5) |- (oALU.lpin 1);
\node [left, xshift=-1cm] (v4) at (oALU.lpin 2){4};
\draw (v4) -- (oALU.lpin 2);
\end{tikzpicture}
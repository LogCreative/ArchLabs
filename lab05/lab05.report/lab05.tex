\documentclass[a4paper,UTF8]{ctexart}
\usepackage{graphicx}
\usepackage{geometry}
\usepackage{xcolor}
\usepackage{amsmath}
\usepackage{enumerate}
\usepackage{caption}
\usepackage{listings}
\usepackage{array}
\usepackage{booktabs}
\usepackage{tikz}
\usetikzlibrary{shapes,arrows}
\usepackage{pgfplots}
\pgfplotsset{compat=1.17}
\usepackage{appendix}
\captionsetup[lstlisting]{labelfont=bf,justification=justified}
\usepackage{multicol}
\setlength{\columnsep}{3em}
\usepackage{float}
\usepackage{circuitikz}

\graphicspath{{img/}}

\usepackage[colorlinks,linkcolor=blue]{hyperref}
\usepackage{bookmark}
\providecommand{\code}[2]{\lstinputlisting[language=#2,caption=\href{run:#1}{\ttfamily #1}]{#1}}
\providecommand{\img}[1]{\includegraphics[width=0.88\textwidth]{#1}}

% listings
\definecolor{grey}{rgb}{0.8,0.8,0.8}
\definecolor{darkgreen}{rgb}{0,0.3,0}
\definecolor{darkblue}{rgb}{0,0,0.3}
\lstset{%
    numbers=left, %行号
    numberstyle=\scriptsize\color{grey},
    showstringspaces=false,
    showspaces=false,%
    tabsize=4,%
    frame=shadowbox,%
    basicstyle={\ttfamily\normalsize},%
    keywordstyle=\color{blue!80!black}\bfseries,%
    identifierstyle=,%
    commentstyle=\color{green!50!blue}\itshape,%
    stringstyle=\color{green!50!black},%
    rulesepcolor=\color{gray!20!white},
    breaklines,
    columns=flexible,
    extendedchars=false,
    %mathescape=true,
    language=verilog,
}

\begin{document}
\title{\normalsize \underline{计算机系统结构实验}\\\LARGE 实验 5 报告\\\vspace*{1em}\normalsize 类MIPS单周期处理器的设计与实现}
\author{李子龙\\ 518070910095}
\date{\today}
\maketitle
\tableofcontents
\clearpage

\section{实验目的}

\begin{enumerate}
    \item 完成单周期的类MIPS处理器
    \item 设计支持16条MIPS指令(add, sub, and, or, addi, andi, ori, slt, lw, sw, beq, j, jal, jr, sll, srl)的单周期CPU
\end{enumerate}

\section{原理分析}

\subsection{指令存储器}

\begin{figure}[h]
    \centering
    \includegraphics[width=\textwidth]{instruction.pdf}
    \caption{指令存储器}
    \label{fig:instr}
\end{figure}

指令存储器通过读取 PC 的值获取需要读取的指令地址。如果读取的内存地址越界,就读取第一个指令。

指令存储器被设定为 64 行。指令存储器的实现如下:

\begin{lstlisting}[caption=InstMemory.v]
module InstMemory(
    input [31:0] readAddress,
    output [31:0] inst
    );

    reg [31:0] instructions [0:63];
    assign inst = instructions[readAddress / 4 < 64 ? readAddress / 4 : 0];

endmodule
\end{lstlisting}

\subsection{顶层模块设计}

\begin{figure}[h]
    \centering
    \includegraphics[width=\textwidth]{struct.pdf}
    \caption{顶层设计}
    \label{fig:top}
\end{figure}

图 \ref{fig:top} 展示了针对 9 条指令的顶层模块设计。而针对 16 条指令,需要对一些输入输出进行扩展。

Ctr 模块会被扩展三个输出信号。
\begin{table}[htbp]
    \centering
    \caption{Ctr 的扩展信号}
    \begin{tabular}{>{\sffamily}cl>{\ttfamily}c>{\ttfamily}c}
        \toprule
        信号 & 描述 & 输出 & 输入\\
        \midrule
        zext & 是否为零扩展 & Ctr & sigext \\
        imm & 是否是立即数命令 & Ctr & ALUCtr \\
        jal & 是否是跳转链接命令 & Ctr & Registers \\
        \bottomrule
    \end{tabular}
\end{table}

扩展信号输出后就可以对于增添的指令进行识别,在实验 3 表 2 的基础上扩增:
\begin{table}[htbp]
    \centering
    \caption{增扩指令}
    \begin{tabular}{>{\sffamily}c>{\ttfamily}c>{\ttfamily}c>{\ttfamily}c>{\ttfamily}c}
        \toprule
        信号 & addi & andi & ori & jal \\
        \midrule
        zext & 0 & 1 & 1 & 0\\
        imm & 1 & 1 & 1 & 0\\
        jal & 0 & 0 & 0 & 1\\
        \bottomrule
    \end{tabular}
\end{table}

针对 ALUCtr 也会进行信号扩增。
\begin{table}[htbp]
    \centering
    \caption{ALUCtr 的扩展信号}
    \begin{tabular}{>{\sffamily}cl>{\ttfamily}c>{\ttfamily}c}
        \toprule
        信号 & 描述 & 输出 & 输入\\
        \midrule
        nop & 是否为空指令 & InstMemory & ALUCtr \\
        jr & 是否为跳转寄存器指令 & ALUCtr & PC \\
        shamt & ALU第一个信号是否是 shamt 位 & ALUCtr & ALU \\
        \bottomrule
    \end{tabular}
\end{table}

添加 \verb"nop" 信号是因为其指令与 \verb"sll" 冲突。
\begin{table}[htbp]
    \centering
    \caption{增广指令}
    \begin{tabular}{>{\sffamily}c>{\ttfamily}c>{\ttfamily}c>{\ttfamily}c>{\ttfamily}c>{\ttfamily}c>{\ttfamily}c|>{\ttfamily}c}
        \toprule
        指令 & op & rs & rt & rd & shamt & funct & ALUCtrOut \\
        \midrule
        nop & 000000 & 00000 & 00000 & 00000 & 00000 & 000000 & 1111\\
        sll & 000000 & 00000 & rt & rd & shamt & 000000 & 1000\\
        srl & 000000 & 00000 & rt & rd & shamt & 000010 & 1001\\
        jr & 000000 & rs & 00000 & 00000 & 00000 & 001000 & 1111\\
        \bottomrule
    \end{tabular}
\end{table}

添加这些数据线后,就可以支持 16 条指令了。并注意对 JAL 进行特殊的处理。

\begin{lstlisting}[caption=Top.v]
    module Top(
        input clk,
        input reset
    );

    reg [31:0] PC;

    always @(posedge clk) begin
        if(reset)
            PC <= 0;
        else begin
            PC <= PC + 4;
        end
    end

    wire [31:0] INST;

    InstMemory instMemory(
        .readAddress(PC),
        .inst(INST)
    );

    wire REG_DST,
        JUMP,
        BRANCH,
        MEM_READ,
        MEM_TO_REG,
        MEM_WRITE;
    wire [1:0] ALU_OP;
    wire ALU_SRC,
        REG_WRITE,
        ZEXT,
        IMM,
        JAL;

    Ctr mainCtr(
        .opCode(INST[31:26]),
        .regDst(REG_DST),
        .jump(JUMP),
        .branch(BRANCH),
        .memRead(MEM_READ),
        .memToReg(MEM_TO_REG),
        .aluOp(ALU_OP),
        .memWrite(MEM_WRITE),
        .aluSrc(ALU_SRC),
        .regWrite(REG_WRITE),
        .zext(ZEXT),
        .imm(IMM),
        .jal(JAL)
    );

    wire [31:0] READ_DATA1;
    wire [31:0] READ_DATA2;
    wire [31:0] OPRAND;
    wire [3:0] ALU_CTR;
    wire ZERO;
    wire [31:0] ALU_RES;
    wire [31:0] READ_DATA;
    wire JR;
    wire SHAMT;

    Registers registers(
        .clk(clk),
        .reset(reset),
        .readReg1(INST[25:21]),
        .readReg2(INST[20:16]),
        .writeReg(JAL ? 5'b11111 : (JR ? INST[25:21] : (REG_DST ? INST[15:11] : INST[20:16]))),
        .writeData(JAL ? PC + 4 : (MEM_TO_REG ? READ_DATA : ALU_RES)), // Jal will jump to PC + 4
        .regWrite(REG_WRITE),
        .readData1(READ_DATA1),
        .readData2(READ_DATA2)
    );

    signext signExt(
        .inst(INST[15:0]),
        .zext(ZEXT),
        .data(OPRAND)
    );

    ALUCtr aluctr(
        .nop(INST == 0 ? 1'b1 : 1'b0), // Avoid nop conflict
        .funct(IMM ? INST[31:26] : INST[5:0]),
        .aluOp(ALU_OP),
        .aluCtrOut(ALU_CTR),
        .jr(JR),
        .shamt(SHAMT)
    );

    ALU alu(
        .input1(SHAMT ? INST[10:6] : READ_DATA1),  
        .input2(ALU_SRC ? OPRAND : READ_DATA2),
        .aluCtr(ALU_CTR),
        .zero(ZERO),
        .aluRes(ALU_RES)
    );

    dataMemory DataMemory(
        .Clk(clk),
        .address(ALU_RES),
        .writeData(READ_DATA2),
        .memWrite(MEM_WRITE),
        .memRead(MEM_READ),
        .readData(READ_DATA)
    );

    always @(negedge clk) begin
    PC <= JR ? ALU_RES : (JUMP ? 
                (PC[31:28] + INST[25:0] << 2) :     // 26 -> 28
                ((BRANCH & ZERO) ? (PC + (OPRAND << 2)) : PC));
    end

endmodule
\end{lstlisting}


\section{仿真结果}

使用了下面的指令文件进行仿真。该指令文件主要的作用是测试所有的运算功能,并在每一个循环对 10 号寄存器 + 1,并存储到 0 号存储单元中,直到其超过刚开始的限制寄存器的存储数字(这里是 4),之后就会进入短循环,不会再对寄存器和存储器进行修改。

\begin{lstlisting}[caption=simple.asm]
nop
lw  $16, 8($0)            # $0 zero register
jal 4
nop
lw  $8, 0($0)
lw  $9, 4($0)
sub $10, $8, $9
and $10, $8, $9
slt $10, $8, $9
or  $10, $8, $9
addi $10, $8, 8
andi $10, $8, -1
ori  $10, $8, -1
sll $10, $8, 1
srl $10, $8, 1
add $10, $8, $9          # final save: += 1
sw  $10, 0($0)
beq $10, $16, 1
jr $31
j 16
\end{lstlisting}

\begin{lstlisting}[caption=mem\_data.mem]
00000001
00000001
00000004
00000000
00000000
00000000
00000000
00000000
00000000
\end{lstlisting}

激励文件加载指令集的时候,要注意 \verb"$readmemb" 是二进制读取,\verb"$readmemh" 是十六进制读取。

\begin{lstlisting}[caption=Basic\_tb.v]
module Basic_tb(

    );

    reg clk;
    reg reset;

    Top Proc(.clk(clk),.reset(reset));

    initial begin
        $readmemb("mem_inst.mem",Proc.instMemory.instructions);
        $readmemh("mem_data.mem",Proc.DataMemory.MemFile,10'h0);
        reset = 1;
        clk = 0;
    end

    always #10 clk = ~clk;

    initial begin
        #80 reset = 0;
    end

endmodule
\end{lstlisting}

\begin{figure}[h]
    \centering
    \includegraphics[width=\textwidth]{figure2.png}
    \caption{16条指令的仿真结果}
    \label{fig:16}
\end{figure}

运算细节见图 \ref{fig:calcde}。所有的运算都得到了正确的结果。

\begin{figure}[h]
    \centering
    \includegraphics[width=\textwidth]{calcdetail.png}
    \includegraphics[width=\textwidth]{memdetail.png}
    \caption{运算细节}
    \label{fig:calcde}
\end{figure}

\section{实验心得}

本次实验调试时间较长,花费近 8 个小时。主要是顶层设计的连线方案比较复杂,对于增扩的指令,需要重新设计端口加以实现。查看波形进行逐个波形的调试,有助于排查故障。这里面对于 ALUCtr 使用了 nop 扩增非常关键,否则会导致空指令执行了某些移位操作导致 0 寄存器非零,从而影响下面数据的读取。本次实验增加了我对单周期处理器设计的理解,对于理解一些原理有很大的帮助。

\end{document}